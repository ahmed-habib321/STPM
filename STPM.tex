\documentclass[]{article}


\input{MyTools}
\usepackage{fancyhdr}
\usepackage{fontspec}
\usepackage{physics}


%\definecolor{cover}{RGB}{230, 194, 24}
\definecolor{cover}{RGB}{51,215,213}
\newfontfamily\Acmefont{Acme-Regular}[
        Path = fonts/Acme/,
        Extension =.ttf
]
\newfontfamily\fancyfont{GreatVibes-Regular}[
        Path = fonts/Great_Vibes/,
        Extension =.ttf
]
\newfontfamily\handfont{PlaypenSans-ExtraBold}[
        Path = fonts/Playpen_Sans/,
        Extension =.ttf
]
\newfontfamily\Garamondfont{EBGaramond-Medium}[
        Path = fonts/Garamond/,
        Extension =.ttf
] 

\begin{document}

\setcounter{section}{0}
\setcounter{equation}{0}
\newpage
\vspace*{\fill}
\begingroup
\thispagestyle{empty}
\begin{center}
    \fontsize{50pt}{0} $\mathbf{Lecture \qquad 6}$
    \par
    \fontsize{50pt}{0} $\mathbf{Linear \qquad System}$
    \par
    \fontsize{50pt}{0} $\mathbf{Of \qquad ODEs}$
\end{center}
\endgroup
\vspace*{\fill}
\newpage   %When making solo lecture remove the comment
%%----------------------------------------------------------------------------------------
%	TITLE PAGE
%----------------------------------------------------------------------------------------
\begingroup
\AddToShipoutPicture*{\put(0,0){\includegraphics[width=\paperwidth,height=\paperheight]{front.jpg}}}
\thispagestyle{empty}
\color{cover}
\begin{center}
    
\end{center}
\endgroup

\newpage
\thispagestyle{empty}

\begingroup
\newgeometry{left=.8in, right=1in, top=.8in}
\begin{center}
    \includegraphics[scale=.5]{collage logo.png}
    \vspace*{1.5cm}
    \par
    {\fontsize{20pt}{30pt}\selectfont
        \textbf{Selected Topics In Pure Mathematics\\(040101401)\\(Fall 2023)}
        \\
        \vspace*{.75cm}
        By
        \vspace*{.75cm}

        Prof. Dr. Mahmoud M. El-Borai

        Department of Mathematics and computer Sciences

        Faculty of Sciences

        Alexandria University
    }

    \vspace*{\fill}
    {\fontsize{10pt}{10pt}\selectfont
    Prepared by : Ahmed M. Habib \& Hazem Hossam
    }
\end{center}
\restoregeometry
\endgroup
\newpage
\tableofcontents
\thispagestyle{empty}
\newpage
\setcounter{page}{1}      %The pages it the start
%\include{BookBody}         %The content of the book

\section{Integro-Partial Differential Equations}
Consider the equation
\begin{equation*}
    \pdv{u(x,t)}{t} = \sum_{|q|\leq m} a_q(x,y,t) D_{x}^{q} u(x,t) + \int_G \sum_{|q|\leq m} b_q(x,y,t) D_{y}^{q} u(x,t) \, dy
\end{equation*}
Where
\begin{itemize}
        \item $G \subset \mathbb{R}^n$ is bounded region with smooth surface
        \item $D_{x}^{q} = \displaystyle \frac{\partial^{|q|}}{\partial x_{1}^{q_1}\partial x_{2}^{q_2}\partial x_{3}^{q_3}\dots \partial x_{n}^{q_n}} $
        \item $|q|=q_1 + q_2 + \dots + q_n$
        \item $x=(x_1,x_2,\dots, x_n)$
        \item $y=(y_1,y_2,\dots ,y_n)$
        \item $dy = dy_1 dy_2 \dots dy_n$
        \item $\int_G = \underbrace{\int\int \dots \int}_{\textit{n}}$
\end{itemize}
\begin{example}
    Consider Integro-Partial differential equation
    \begin{equation}
        \begin{cases}
                \displaystyle \pdv{u(x,t)}{t}=\pdv[2]{u(x,t)}{x} + \int_{a_1}^{b_1} K_1(x,y,t) \pdv{u(y,t)}{y} dy + \int_{a_2}^{b_2}K_2(x,y,t)u(y,t) \, dy
                \\
                u(x,0)=\phi(x)
        \end{cases}
    \end{equation}
    And $\phi(x)$ , $K_1$ , $K_2$ are bounded and continuous known functions on $(-\infty,\infty)$
    \\
    Consider the Cauchy problem 
    \begin{equation}
        \begin{cases}
                \displaystyle \pdv{u(x,t)}{t}=\pdv[2]{u(x,t)}{x} + V(x,t)
                \\
                u(x,0)=\phi(x)
        \end{cases}
    \end{equation}
    The solution of this problem is given by 
    \begin{equation}
        u(x,t) = \underbrace{\frac{1}{2\sqrt{\pi t}}\int_{-\infty}^{\infty}e^{\textstyle -\frac{{(x-\xi)}^2}{4t}} \phi(\xi)d\xi}_{= \psi(x,t)} 
    + \int_{0}^{t} \int_{-\infty}^{\infty}\frac{1}{2\sqrt{\pi (t-\theta)}} e^{\textstyle -\frac{{(x-\xi)}^2}{4(t-\theta)}} V(\xi,\theta)d\xi d\theta
    \end{equation}
    (More information about this problem in the end of the lecture)
    \\
    Comparing problem (1) with (2) we get that  
    \[
        V(x,t) = \int_{a_1}^{b_1} K_1(x,y,t) \pdv{u(y,t)}{y} dy + \int_{a_2}^{b_2}K_2(x,y,t)u(y,t) \, dy
    \]
    Now substitute for $u$ from (3)
    \begin{align*}
        V(x,t) =& \int_{a_1}^{b_1} K_1(x,y,t) \pdv{\psi(y,t)}{y} dy 
                \\
                &+ \int_{a_1}^{b_1} K_1(x,y,t) \pdv{}{y} \int_{0}^{t} \int_{-\infty}^{\infty}\frac{1}{2\sqrt{\pi (t-\theta)}} e^{\textstyle -\frac{{(y-\xi)}^2}{4(t-\theta)}} V(\xi,\theta)d\xi d\theta dy 
                \\
                &+ \int_{a_2}^{b_2}K_2(x,y,t) \psi(y,t) \, dy
                \\
                &+ \int_{a_2}^{b_2}K_2(x,y,t) \int_{0}^{t} \int_{-\infty}^{\infty}\frac{1}{2\sqrt{\pi (t-\theta)}} e^{\textstyle -\frac{{(y-\xi)}^2}{4(t-\theta)}} V(\xi,\theta)d\xi d\theta \, dy
    \end{align*}
    Put $K_1 = 0 $ to make it a little simpler
    \[
        V(x,t) = \psi^*(y,t) + \int_{a_2}^{b_2}K_2(x,y,t) \int_{0}^{t} \int_{-\infty}^{\infty}\frac{1}{2\sqrt{\pi (t-\theta)}} e^{\textstyle -\frac{{(y-\xi)}^2}{4(t-\theta)}} V(\xi,\theta)d\xi d\theta \, dy    
    \]
    Now we have a volttera integral equation that can be solved by the integral equation methods like successive approximation
\end{example}


\setcounter{equation}{0}
\begin{enrichment*}{}
        \begin{center}
                \textbf{Remember From PDE Course(Extra Information)}
        \end{center}
    \end{enrichment*}


\subsection{Cauchy In-Homogeneous Problem}
Also known as Heat with a source Cauchy problem
\\
Consider the in-homogeneous heat equation on the whole line
\begin{equation}
    \begin{cases}
        \displaystyle \frac{\partial u\left(x,t \right)}{\partial t} = c^2 \frac{\partial^2 u(x,t)}{\partial x^2} + f(x,t),\quad c\neq 0,\quad-\infty<x<\infty,\quad t>0
        \\
        u\left(x,0 \right) = \phi\left(x\right)
    \end{cases}
\end{equation}
Where $f(x, t)$ and $\phi(x)$ are arbitrary given functions. 
\\$f(x, t)$ is called the source term, and it measures the physical effect of an external heat source.
\par
From the superposition principle, we know that the solution of the in-homogeneous equation can
be written as the sum of the solution of the homogeneous equation, and a particular solution of the
in-homogeneous equation. 
\\
Thus we can break problem (1) into the following two problems
\begin{equation}
    \begin{cases}
        \displaystyle \frac{\partial u_h\left(x,t \right)}{\partial t} = c^2 \frac{\partial^2 u_h(x,t)}{\partial x^2}
        \\
        u_h\left(x,0 \right) = \phi\left(x\right)
    \end{cases}
\end{equation}
\begin{equation}
    \begin{cases}
        \displaystyle \frac{\partial u_p\left(x,t \right)}{\partial t} = c^2 \frac{\partial^2 u_p(x,t)}{\partial x^2}+ f(x,t)
        \\
        u_p\left(x,0 \right) = 0
    \end{cases}
\end{equation}
Obviously, $u = u_h + u_p$ will solve the original problem (1).
\\
We have solved problem (2) using Poisson formula which is
\begin{equation}
    u_h(x,t) = \int_{-\infty}^{\infty}S(x-y,t) \phi(y)dy        
\end{equation}
Where $S(x,t)$ is the heat kernel and it's equal to $\displaystyle \frac{\textstyle e^{-\frac{x^2}{4tc^2}}}{2\sqrt{\pi tc^2}}$.
\\
Notice that the physical meaning of expression (4) is that the heat
kernel averages out the initial temperature distribution along the entire rod.
\par
Since $f(x, t)$ plays the role of an external heat source, it is clear that this heat contribution must be averaged out too. 
But in this case one needs to average not only over the entire rod, but over time as well, 
since the heat contribution at an earlier time will effect the temperatures at all later times. 
We claim that the solution to (3) is given by
\begin{equation}
    u_p(x,t) = \int_{0}^{t}\int_{-\infty}^{\infty}S(x-y,t-s) f(y,s)dyds
\end{equation}
Combining (4) and (5) we obtain the solution to the IVP (1)
\begin{equation}
    u(x,t)  = \int_{-\infty}^{\infty}S(x-y,t) \phi(y)dy + \int_{0}^{t}\int_{-\infty}^{\infty}S(x-y,t-s) f(y,s)dyds
\end{equation}
Now substitute the heat kernel
\[
    u(x,t) = \frac{1}{2\sqrt{\pi tc^2}}\int_{-\infty}^{\infty}e^{\textstyle -\frac{{(x-y)}^2}{4tc^2}} \phi(y)dy + \frac{1}{2\sqrt{\pi (t-s)c^2}} \int_{0}^{t} \int_{-\infty}^{\infty}e^{\textstyle -\frac{{(x-y)}^2}{4(t-s)c^2}} f(y)dyds            
\]
We now need to show that (6) indeed solves problem (1) by direct substitution. 
\\
Since we have solved the homogeneous equation before, it suffices to show that $u_p$ solves problem (3). 
\par
By differentiating (5) with respect to t gives
\[
    \frac{\partial u_p}{\partial t} = \int_{-\infty}^{\infty}S(x-y,0) f(y,t) dy  + \int_{0}^{t}\int_{-\infty}^{\infty}\frac{\partial}{\partial t}S(x-y,t-s) f(y,s)dyds    
\]
\begin{enrichment*}{}
    Note that the heat kernel solves the heat equation and has the Dirac delta function as its initial
    means that $S_t =c^2 S_{xx}$ and $S(x-y,0) = \delta(x-y)$
    \\
    When integrating the Dirac Delta function we would get
    \[
        \int_{-\infty}^{\infty}\delta(x-y) dy = 1
    \]
    If we have another function $f(y,t)$ multiplied to the Dirac Delta function and integrating them we would get
    \[
        \int_{-\infty}^{\infty}\delta(x-y) f(y,t) dy = f(x,t)\int_{-\infty}^{\infty}\delta(x-y) dy = f(x,t) 
    \]
\end{enrichment*}
\begin{align*}
    \frac{\partial u_p}{\partial t} &= \int_{-\infty}^{\infty}\delta(x-y) f(y,t) dy  + \int_{0}^{t}\int_{-\infty}^{\infty}c^2 \frac{\partial^2}{\partial x^2}S(x-y,t-s) f(y,s)dyds
    \\
    &= f(x,t) + c^2 \frac{\partial^2}{\partial x^2} \int_{0}^{t}\int_{-\infty}^{\infty}S(x-y,t-s) f(y,s)dyds
    \\
    &= f(x,t) + c^2 \frac{\partial^2 u_p}{\partial x^2}
\end{align*}
Which shows that $u_p(x,t)$ solves the in-homogeneous heat equation. It is also clear that
\[
\lim_{t \to 0} u_p(x,t)  = \lim_{t \to 0}\int_{0}^{t} \int_{-\infty}^{\infty}S(x-y,t-s) f(y,s)dyds = 0
\]
Therefore $u_p(x,t)$ indeed solves problem (3) which finishes the proof that (6) solves the original IVP (1).
\begin{enrichment*}{Duhamel's principle}
    If one can solve an initial value problem for a homogeneous linear differential equation then an in-homogeneous linear differential equation can be solved as well.
\end{enrichment*}
%%%%%%%%%%%%%%%%%%%%%%%%%%%%%%%%%%%%%%%%%%%%%%%%%%%%%%%%%%%%%%%%%%%%%%%%
%%%%%%%%%%%%%%%%%%%%%%                     %%%%%%%%%%%%%%%%%%%%%%%%%%%%
%%%%%%%%%%%%%%%%%%        End of the book       %%%%%%%%%%%%%%%%%%%%%%%
%%%%%%%%%%%%%%%%%%%%%%                     %%%%%%%%%%%%%%%%%%%%%%%%%%%%
%%%%%%%%%%%%%%%%%%%%%%%%%%%%%%%%%%%%%%%%%%%%%%%%%%%%%%%%%%%%%%%%%%%%%%%
\begingroup
\newpage
\AddToShipoutPicture*{\put(0,0){\includegraphics[width=\paperwidth,height=\paperheight]{back.jpg}}}
\thispagestyle{empty}
\color{cover}
\newgeometry{left=.8in, right=1in, top=.8in, bottom=.8in}
\begin{center}
    {
        \fontsize{20pt}{0}\handfont \color{white}
        Selected Topics In Pure Mathematics
    }
\end{center}

\vspace*{.5cm}
{\fontsize{15pt}{0} \Garamondfont \selectfont
    In this book embark on a captivating journey through the intricate web of pure mathematics, climate change models, and dynamical systems. Delve into the depths of stability theory, where the delicate balance between order and chaos shapes the world around us.
    \\Explore the mystique of climate change models, unraveling the secrets hidden within the Earth's complex ecosystems.
    \par
    \vspace*{1cm}
    At the heart of this compelling narrative are the groundbreaking methods of solving nonlinear systems, where successive approximation and the revolutionary Adomian decomposition method converge.
    \\Witness the fusion of theory and application as mathematicians and scientists collaborate to decipher the enigmas of our changing climate.
    \par
    \vspace*{1cm}
    Through the lens of rigorous mathematical analysis,
    this book illuminates the profound connections between stability and climate change, offering fresh perspectives on the challenges that lie ahead.
    \\This book is a testament to the power of mathematics, revealing its crucial role in understanding the world and shaping our future. Prepare to be enthralled, enlightened, and inspired as you explore the elegant symphony of mathematics, climate science, and dynamical systems
}

\par
\vspace*{5cm}
{\fontsize{20pt}{0}\fontspec{Arial}
    Author : \\\\
    \color{white} \fancyfont Prof. Dr. Mahmoud M. El-Borai
    \\ \\ \fontspec{Arial}
    \color{cover}Prepared by :\\\\
    \color{white} \fancyfont Ahmed Mohamed Habib \& Hazem Hossam
    \\ \\\fontspec{Arial}
    \color{cover} Cover Designed by :\\\\
    \color{white} \fancyfont Hazem Hossam
}
\restoregeometry
\endgroup          %The pages in the End
\end{document}